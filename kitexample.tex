%% LaTeX-Beamer poster template for KIT design
%% by Erik Burger, Christian Hammer
%%
%% version 1.2
%%
%% mostly compatible to KIT corporate design v1.2
%% http://www.uni-karlsruhe.de/download/uka/Gestaltungsrichtlinien_komplett.pdf
%%
%% Problems, bugs and comments to
%% burger@kit.edu

\documentclass{beamer}

%% Fill in the page size here. If the proportions of the fonts
%% are not satisfactory, change the scale parameter
\usepackage[orientation=portrait,size=a0,scale=1.4]{beamerposter}
\usepackage{blindtext}

\usepackage{comment}
\mode<presentation>{\usetheme{kitposter}}

\title[Short title]{Full title:\\[.4em] With all details}
\subtitle{Something for XYZ 2009}
\author{Firstname1 Lastname1, Firstname2 Lastname2}

\institute{\textbf{Steinbuch Centre for Computing\\[.2em]
Computational Science and Mathematical Methods }\\[.4em]
Hermann-von-Helmholtz-Platz 1\\[.2em]
76344 Eggenstein-Leopoldshafen\\[.4em]
\href{https://www.scc.kit.edu/en/index.php}{www.scc.kit.edu}}

\selectlanguage{english}


\begin{document}
% change the following line to "ngerman" for German style date and logos


\begin{frame}
	\blindtext

	\vfill
	\begin{minipage}{0.45\textwidth}                % THE LEFT HALF OF THE POSTER
		
		This is insanely ugly but I don't know how to make it pretty.
		
		\begin{block}{This is a rounded box}
			Some text
			\begin{itemize}
				\item this
				\item that
			\end{itemize}
		\end{block}
		Remember:
		\begin{align}
			x^2+y^2=z^2
		\end{align}
		\begin{block}{Another block}
			The quick brown fox jumps over the lazy dog.	
		\end{block}
		\begin{block}{Another block}
			The quick brown fox jumps over the lazy dog.	
		\end{block}
	\end{minipage}
	\hfill
	\begin{minipage}{0.45\textwidth}                % THE LEFT HALF OF THE POSTER
		\blindtext
		\begin{figure}
			\centering
			\includegraphics[width=0.7\linewidth]{logos/kitlogo_en_cmyk}
			\caption{{\color{kit-blue100} Why does it say 'Abbildung' in German?}}
			\label{fig:kitlogoencmyk}
		\end{figure}
		\begin{block}{Blindtext}
			\blindtext
		\end{block}
	\end{minipage}
	\vfill
	\blindtext
\end{frame}
\end{document}